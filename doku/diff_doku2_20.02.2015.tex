	\subsubsection{Sensorik}
	Sensorisch erfasst werden die Helligkeit und die Bodenfeuchtigkeit. 
	Um eine Pflanze mit Wasser zu versorgen reicht es zu wissen wie feucht der Boden ist. 
	Aber um Rücksicht auf die Bewohner einer Wohnung zu nehmen ist zusätzlich ein Helligkeitssensor verbaut, dieser soll dafür sorgen, dass bei Dunkelheit im Raum eine "Nachtruhe" gehalten wird.
	Der Bodenfeuchtesensor bestimmt den Wassergehalt des Bodens über einen Widerstand zwischen zwei Zinken einer Messgabel. 
	Leider zeigte sich das nach nur 48 Stunden Dauermessung die Gabel erhebliche Korrosion erlitten hat.
	Außerdem sieht die Vorschaltung keine Abschaltung des Messprozesses vor noch eine Umpolung der Gabel.
	Deswegen muss die gesamte Vorschaltung stromlos geschaltet werden, um das Auflösen des Sensors zu verlangsamen. 
	
	\subsubsection{Stromversorgung}
	Es werden drei Spannungslevel in der Gießanlage benötigt. 
	Einmal 12 Volt zum Betreiben der Pumpe, zum zweiten 5 Volt für den Microcontroller und die Sensorik und  zuletzt  3,3 Volt für das XBee-Modul. 
	Deswegen wird das System so ausgelegt, dass 12 Volt Eingangsspannung aufnimmt. 
	Das Arduino Nano erzeugt daraus mit Hilfe zweier Linearwandlern die Spannungen 5 Volt und 3,3 Volt.
	



