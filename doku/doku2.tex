\documentclass[]{IEEEtran}
\usepackage[utf8]{inputenc}		%UTF8 input file
\usepackage[T1]{fontenc}		
\usepackage[ngerman]{babel}
\usepackage{amsmath, amssymb}
\usepackage{eurosym}
\usepackage{hyperref}
%Für Fußzeilten
%\usepackage{stfloats}
%Bilder
\usepackage{graphicx}
\usepackage{epstopdf}
%Zeilenabstände
\usepackage{setspace}


%Für Quellcode
\usepackage{listings}


\title{Pflanzengießanlage}
%\subtitle{DIY14Pflanze}
\date{\today}
\author{ Stefan Schubäck, \and Matthias Nagl,\and Christoph Hofbauer, \and Dmitrii Cetvericov, \and Markus Fischer-Has,}

%\renewcommand{\iedlistdecl}{\settowidth{\labelwidth}{Hello}}

\begin{document}


	\maketitle
%	\tableofcontents

\begin{abstract}
Auch unter den Studenten gibt es den einen oder anderen mit einem grünen Daumen, der seine Wohnung durch ein paar Pflanzen aufgewertet hat. 
Wenn nicht das ständige Gießen wäre. 
Vor jedem längeren Urlaub stellt sich die Frage, was machen mit den Pflanzen? 
Den Nachbarn fragen und hoffen er ist da und verlässlich? 
Die Pflanzen \glqq Absaufen\grqq \ lassen und hoffen, dass die sich das Wasser einteilen? 
Beides keine zufriedenstellende Lösung. Wieso lässt man sich die Pflanze nicht selber gießen?
Aus diesen Überlegungen ist die Idee entstanden, eine Pflanze mit Sensoren auszustatten, die uns über den aktuellen Zustand informieren. 
Im Folgenden wird der Aufbau zweier Lösungen vorgestellt. 
Der erste Ansatz basiert auf dem Arduino-System, dass einen einfachen Einstieg in Umgang mit Mikrocontroller ermöglicht. 
Der zweite Ansatz baut auf ähnliche Hardware ohne die Verwendung der Arduino Software. 
Dadurch wird die Verwendung von Sleep"~Modi und Interrupts möglich, was zu einer Low"~Energy Variante des Gießsystems führt.  

\end{abstract}

\section{Eingrenzung}
Es lassen sich viele Faktoren finden, die einen Einfluss auf das Wohlbefinden eine Pflanze haben. 
Hierunter fallen die Bodenfeuchtigkeit im Topf der Pflanze, die durchschnittliche Sonneneinstrahlung, der Luftqualität, der Nährstoffgehalt des Bodens sowie die Temperatur an der Wurzel der Pflanze. 
Diese nicht abschließende Aufzählung zeigt, dass eine Eingrenzung der zu untersuchenden Faktoren notwendig ist um das Projekt in der gegeben Zeit fertigzustellen.
Auch in Hinblick des Einsatzgebietes gibt eine Vielfalt von unterschiedlichen Möglichkeiten wie z.B. Einpflanzenbetrieb oder Mehrpflanzenbetrieb sowie Indoor oder Outdoor. 
Im Rahmen des Projekts werden demnach folgende Eingrenzung vorgenommen:

\begin{itemize}
	\item Die Bodenfeuchtigkeit wird abgeleitet aus der Widerstandsänderung einer Messgabel die sich im Erdreich der Pflanze befindet. Die Widerstandsänderung ergibt sich aus der Änderung des Wassergehalts der Erde.
	\item Das System soll für eine alleinstehende Pflanze aufgebaut werden. 
	\item Das Einsatzgebiet wird im Innenbereich sein, dadurch fallen Einschränkungen bzgl. Witterungs"-beständigkeit weg.
	\item Die Stromversorgung wird über ein Netzteil bewerkstelligt. 
	\item Das System soll über eine drahtlose Verbindung einstellbar sein. Diese Einschränkung gilt nur für die Arduino Variante, da in der Sparversion auf die Kommunikation verzichtet wird. Hier wird die Einstellung direkt im Code vorgenommen.

\end{itemize}

Weitere Ideen wie z.B. eine autarke Lösung mit Batteriebetrieb, ein Mehr"~Pflanzenbetrieb oder ein System für den Garten wurden zwar diskutiert, aber wegen dem erhörten Zeitaufwand und begrenzten Budgets nicht weiter verfolgt. 
Im folgenden wird die \emph{Arduino Version} ausführlich vorgestellt. 
Die \emph{Sparversion} wird im Anschluss erläutert, da viele der Funktionen grundsätzlich identisch sind wird darauf verzichtet diese erneut auszuformulieren. 

%section

\section{Arduino Variante}

In Abbildung \ref{fig-SchemaAufbau} 


\end{itemize}
	\begin{figure}[!h]
	\centering
	\includegraphics[width=0.8\linewidth]{bilder/Bild_Aufbau.eps}	
	\caption{Schematischer Aufbau der Gießanlage}
	\label{fig-SchemaAufbau}
	\end{figure}

Im Folgenden beschreiben wir das Vorgehen für das Gießsystem basierend auf der Arduino Variante. 
Es wird hierzu auf folgende Bereiche eingegangen:

\begin{itemize}
	\item Zu Beginn beschreiben wir verschiedenen Möglichkeiten des Wassertransports zur Pflanze.
	\item Im weiteren stellen wir das Vorgehen zur Erstellung des Gehäuses für die Elektronik dar.
	\item In Abschnitt Elektronik werden wir auf den Aufbau der Platine und die hiermit verbunden Bereiche \emph{Sensorschaltung}, \emph{Stromversorgung} und \emph{Kommunikation} eingehen.
	\item Anschließend gehen wir auf das Programm ein, mit dem entschieden wird wann der Zeitpunkt erreicht ist um die Pflanze zu gießen.
	\item Abschließenden wir eine Kostenplan vorgestellt. 
	


\subsection{Wassertransport}
Um den Einsatzbereich so flexibel wie möglich zu halten haben wir uns für ein Pumpensystem entschieden. 
Dadurch ist die Anordnung der Pflanze zum Wassertank nicht relevant. 
Der hiermit einhergehen erhöhte Strombedarf ist zu verkraften, da unser System nicht auf Batterien angewiesen ist.

Für den Wassertransport haben wir uns für eine Zahnradpumpe entschieden.
Die Zahnradpumpe gehört zu der Gruppe der rotierenden Verdränger Pumpen 
\footnote{\href{http://www.ksb.com/Kreiselpumpenlexikon\_de/Pumpenlexikon/1563382/verdraengerpumpe.html}{www.ksb.com/Kreiselpumpenlexikon\_ \\ de/Pumpenlexikon/1563382/verdraengerpumpe.html}}.
Das Fördermedium wird hierbei zwischen zwei Zahnrädern in einem in sich geschlossenen Volumenbereich gefördert.
Die Bauweise dieser Pumpe ermöglicht zudem einen selbstsaugenden Betrieb. 
Dies bedeutet, dass diese Pumpe in der Lage ist Gase zu transportieren und somit einen Unterdruck in der Zuleitung zu erzeugen, der ausreicht, um das Fördermedium (in unserem Fall Wasser) anzusaugen. 
Diese Eigenschaft war schlussendlich ausschlaggebend, dass sich die Zahnradpumpe gegenüber den anderen Lösungen durchgesetzt hat.
Als Alternativen wurden Ventile und Kreiselpumpen angedacht.
Die Kreiselpumpe konnte trotz dem geringeren Stromverbrauch und geringerem Geräuschpegel nicht durchsetzen. 
Noch weniger Strom und Lärm verursachen Ventile, die aber auf gespeicherte Energie angewiesen sind.  
Entweder durch Druck im Wassertank oder durch Erhöhung des Tankes über den Ausfluss. 
Auf Grund der Wahl der Zahnradpumpe kann ein 4\,mm Schlauch zur Förderung des Wassers genutzt werden, der nahezu beliebig verlegt werden kann.  
Ebenso ist das Gefäß frei wählbar, für das Testsystem haben wir eine 1,5\,Liter Flasche verwendet. Im Dauerbetrieb wird ein fünf-Liter"~Weinballon verwendet. 
In der Pflanze wird der Schlauch mit einem durchbohrten Kantholz in der Pflanzenerde befestigt.

	
\begin{table}
	\centering
		\onehalfspacing
	\footnotesize
	\caption{Vergleich Wasserpumpen und Ventil}
	\label{Vergleich zwischen Wasserpumpen und Ventil}
		\begin{tabular}{|l|lll|}
		\hline
		\textit{Eigenschaft} & \textit{Zahnradpumpe} & \textit{Kreiselpumpe} & \textit{Ventil} \\
		\hline
		Selbstsaugend	&ja	&nein &nein\\		
		Lautstärke		&sehr laut	&mittel laut	&leises Klacken\\
		Stromverbauch	&@12V 2,8A	&@12V 0,6A	&@12V 80mA\\
		Förderleistung	&gering		&groß		&keine eigene\\
		Preis			&2,95 \euro	& 2,95 \euro	&	4,95 \euro\\
		\hline		
		\end{tabular}
		
\end{table}	
	
	
	
\subsection{Gehäuse}
	Das Gehäuse wurde so konzipiert, dass es möglichst klein ist aber genügend Platz für die Elektronik bietet.
	Im Gehäuse verbaut sind:
\begin{itemize}
	\item ein LCD-Display zur Anzeige der Gießparameter (aktelle Feuchtigkeit, Gießintervall \dots)
	\item zwei Taster zur Steuerung der Displayanzeige und zum manuellen Gießen
	\item der Photowiderstand zur Messung der Helligkeit (platziert zwischen den Tastern)
	\item Stromschluss auf der linken Seite
	\item Ausgänge für die Pumpe und den Feuchtigkeitssensor auf der rechten Seite

\end{itemize}	

	\begin{figure}[!h]
	\centering
	\includegraphics[width=0.8\linewidth]{bilder/_boxFron1.jpg}	
	\caption{Gehäuse Frontansicht}
	\label{fig-Gehäuse}
	\end{figure}
	
Das Gehäuse wurde in FabLab Erlangen mit einem Lasercutter gefertigt. 
Für das Design des Gehäuses wurde der BoxMaker\footnote{ \href{http://boxmaker.connectionlab.org/}{http://boxmaker.connectionlab.org/}} verwendet. 
Das Gehäuse besteht aus grünen 3\,mm dicken Acrylglas.
	
\subsection{Elektronik}
Im folgenden wird der Aufbau der Version~1.0 der Arduino Variante vorgestellt. 
Während des Aufbaus, vor allem aber während der Testphase sind ein paar Probleme aufgetreten die dazu geführt habe, dass eine neue Version~1.1 erstellt werden musste. 
Diese neue Version ist jedoch aus zeitlichen Gründen noch nicht komplett aufgebaut. 
Ein neues Layout wurden bereits erstellt und die Software ist so angepasst worden, dass diese ohne größere Änderungen übernommen werden kann. 
Daher wird im folgenden die Funktionsweise auf Basis der Version~1.0 erläutert, an gegebener Stelle wird auf die Anpassungen eingegangen, die bereits umgesetzt wurden bzw. noch in Planung sind. 

			
\subsubsection{Sensorik} \label{sensorik}
Für den Helligkeitssensor wird ein einfacher Photowiderstand verwendet, der über einen Spannungsteiler an einem der Analogen Pins des Arduino Bords angeschlossen ist. Der Pin verfügt über einen 10\,Bit AD Konverter und gibt demnach einen Integerwert von 
0\,-\,1023 zurück.
Der Bodenfeuchtigkeitssensor bestimmt den Wassergehalt des Bodens über eine Widerstandsmessung zwischen den zwei Zinken einer Messgabel. 
Je mehr Wasser im Erdreich vorhanden ist, desto kleiner ist der gemessene Widerstand im Boden.

\begin{figure}[!h]
	\centering
	\includegraphics[width=0.8\linewidth]{bilder/_feuchteSensor1.jpg}
	\caption{Feuchtigkeitssensor mit Vorschaltung}
	\label{fig-SensorVorschaltung}
\end{figure}
Der Feuchtigkeitssensor benötigt keine weiteren Schaltelemente, da er über eine eine Vorschaltung verfügt in der ein Spannungsteiler bereits verbaut ist. 
Abbildung \ref{fig-SensorVorschaltung} zeigt den verwendeten Sensor und die Vorschaltung. 
Es ist möglich sowohl den Analogen Wert, oder ein Digitales Signal auszuwerten. 
Das Digitale Signal liefert einen Null"-wert solange ein Grenzwiderstand nicht überschritten wird. 
Über ein  Potentiometer(siehe Abbildung \ref{fig-SensorVorschaltung}) lässt sich diese Grenze einstellen. 
Wegen des schlechten Zugangs zum Potentiometer im eingebautem Zustand wird der Digitale Output nicht verwendet, sondern der Analoge Messwert selbst ausgewertet und mit einer Variablen im Mikrocontroller abgeglichen.
		
\emph{Anpassung zu Version~1.1:}
Leider zeigte sich, dass nach nur 48 Stunden Dauermessung die Gabel erhebliche Korrosion erlitten hat, Abbildung \ref{fig-SensorVergleich} zeigt dies deutlich.
Die Vorschaltung sieht keine Abschaltung des Messprozesses vor noch eine Umpolung der Gabel. 
Deswegen muss die gesamte Vorschaltung stromlos geschaltet werden, um das Auflösen des Sensors zu verlangsamen. 
Hierfür haben, wir für die Version~1.1, eine Transistorschaltung für den Sensor eingefügt. 
Über diese Schaltung wird die Vorschaltung des Feuchtigkeitssensor stromlos geschaltet und der Sensor ist nur für eine Messung unter Strom. 
Eine Umpollung der Messgabel ist auch nicht möglich, dies kann jedoch relativ einfach dadurch gelöst werden, indem der Sensor alle paar Wochen \emph{manuell} umgepolt wird. 
Hierfür muss lediglich die Messgabel vom Verbindungskabel abgesteckt werden und um \begin{math}180^{\circ}\end{math} gedreht wieder verbunden werden.

\begin{figure}[!h]
	\centering
	\includegraphics[width=0.8\linewidth]{bilder/_fechtesensorVergleich0.jpg}
	\caption{Vergleich neuer Sensor und Sernsor mit 48h Dauerbetrieb}
	\label{fig-SensorVergleich}
\end{figure}

\subsubsection{Kommunikation}
Jede Pflanze braucht unterschiedlich viel Wasser. 
Genauso hat die Zusammensetzung der Erde einen Einfluss auf den gemessenen Widerstand des Feuchtigkeitssensors.
Ziel der Kommunikation soll es deswegen sein, dass Gießsystem im laufenden Betrieb zu kalibrieren. 
Wir haben uns für eine drahtlose Kommunikation, auf Basis des XBee Standard, entschieden. 
Das XBee Modul wurde direkt, d.h. ohne Verwendung eines \emph{Arduino XBEE"~Shields}, mit dem Arduino verbunden. 
Hierfür ist es notwendig einen Pegelwandlung von 5\,V auf 3,3\,V vorzunehmen, da die Logik-Pins des XBee Moduls nicht mit 5\,V betrieben werden können. 
%%%%%%
Abbildung \ref{fig-Pegel} zeigt die Umsetzung der Pegelschaltung. Die Lables \emph{TXD} und \emph{RXD} repräsentieren die jeweiligen Anschlüsse am Arduino Micro. In dieser Schaltung lassen sich drei Zustände erkennen\footnote{http://www.adafruit.com/datasheets/an97055.pdf; Seite 10f}
\begin{itemize}
\item Wird keiner der beiden Seiten (Mikrocontroller bzw. XBee) auf GND gezogen, so werden die Eingänge durch ihre  Pull-Up 	Widerstände auf 5\,V bzw. 3,3\,V gezogen. Dadurch ist das Spannungsgefälle \emph{Gate"~zu"~Source} gleich Null, da an beiden 	Anschlüssen 3,3\,V anliegen. Der Transistor ist nicht leitend.
\item Wird am XBee (DOUT) ein LOW Signal angelegt, so steigt das Spannungsfälle \emph{Gate"~zu"~Source} an und der Transistor wird leitfähig, was dazu führt das auch die Pins am Mikrocontroller(RXD) auf LOW gezogen werden.
\item Wird hingen auf der Seite des Mikrocontroller (TXD) ein LOW Signal angelegt, so wird über die die Diode im Transistor das Spannungspotenzial der Source solange reduziert, bis das Spannungsgefälle \emph{Gate"~zu"~Source} groß genug wird, damit der Transistor leitfähig wird. Sobald dieser Grenzwert überschritten wird, wird auch DOUT am Xbee auf LOW gezogen.
\end{itemize}

Bei der Pegelschaltung ist uns in Version~1.0 ein Fehler unterlaufen, weswegen wir keine Verbindung mit ein Computer aufbauen konnten. Dieser Fehler ist für die Version~1.1 behoben.

\begin{figure}
	\centering
	\includegraphics[width=0.8\linewidth]{bilder/v1SchaltplanXbee.jpg}
	\caption{Pegelwandler mit Small-Signal-Transistor BSS138W }
	\label{fig-Pegel}
\end{figure}
		
Die zweite Schnittstelle mit dem Menschen wird über ein Display ermöglicht. 
Hier lässt sich über einen Taster die einzelnen Variablen mit ihren aktuellen Werten Überprüfen. 
Bei dem Display Handelt es sich um ein 2 Zeiliges LCD"~Display mit jeweils 16 Zeichen. 
Die Kommunikation zwischen Display und Mikrocontroller wird über eine I2C"~Schnittstelle bewerkstelligt. 
Hierfür haben wir die frei verfügbare \mbox{LiquidCrystal\_I2C} Libary von fderbrabander verwendet.\footnote{\href{https://github.com/fdebrabander/Arduino-LiquidCrystal-I2C-library}{https://github.com/fdebrabander/Arduino-LiquidCrystal-I2C-library}}

\begin{figure*}[!h]
	\centering
	\includegraphics[width=0.9\linewidth]{bilder/v1SchaltplanMicro0.JPG}
	\caption{Schaltpaln V1.0 mit Arduino Micro}
	\label{fig-Schaltplanv1.0}
\end{figure*}

	
\subsubsection{Stromversorgung}

Eine Stromversorgung über USB ist nicht möglich, da die Pumpe in Voll"-last 12\,V und außerdem 2,8\,A benötigt. 
Deswegen muss auf eine leistungsfähigere Energiequelle gesetzt werden.
Wir entschieden uns für Netzteile mit 12\,V Ausgangs"-spannung um die Pumpe direkt anschließen zu können. 
Um den Strom der Pumpe zu begrenzen haben wir einen \begin{math}5~\Omega\end{math} Lastwiderstand in Reihe geschaltet.
Dies führt zu geringerer Leistungsaufnahme und deutlichen Geräusch"-minderung.
Die Ansteuerung der Pumpe über dem Mikrocontroller ist über eine Transistorschaltung gelöst (Ziffer 1 in Abbildung \ref{fig-Schaltplanv1.0}).
Es ist zu überlegen, den Transistor und den Mikrocontroller über eine Schutzdiode über die Anschlüsse \emph{VENTIL-1} und \emph{VENTIL-2} vor Überspannung zu schützen, die beim Abschalten der Pumpe auftreten können. 
 

\subsection{Programm}
	

	Die Aufgaben des Mikrocontrollers sind die folgenden:
		\begin{itemize}
			\item Messung der Feuchtigkeit
			\item Messung der Helligkeitssensor
			\item Vergleich der Messwerte mit den festgelegten Grenzen
			\item Ansteuerung der Pumpensystem
			\item Ausgabe der wichtigen Variablen auf dem Display über Taster
			\item Manuelles Gießen über Taster
			\item Display Hintergrundbeleuchtung abschalten nach 60  Sekunden
			\item Übertragen/Empfangen von Einstellungen über xBee Verbindung
		\end{itemize}
		
	In der ersten Version wurden die Messungen andauernd durchgeführt. 
	Wie aber bereits oben erläutert hat dies zu einer starken Korrosion am Feuchtigkeitssensor geführt. 
	Aus diesem Grund wurde für die Version~1.1 einige Änderungen eingeführt. 
	Anstelle bei jedem Aufruf der Loop() Methode eine Messung durchzuführen wird nur noch alle 4 Stunden eine Messung vorgenommen. 
	Ist das Intervall abgelaufen wird der Feuchtigkeitssensor über eine Transistorschaltung angeschaltet und anschließend ausgelesen. 
	Dadurch ist der Sensoren nur noch für einen kurzen Zeitraum unter Strom, was die Lebensdauer des Feuchtigkeitssensor verlängert.
		
	
	%\subsection{Setuptool}
	Da die Erde in jeder Pflanze eine andere Zusammensetzung hat und jede Pflanze unterschiedlich viel Wasser benötigt ist es notwendig die Grenzwerte für die Feuchtigkeit und die Wassermenge einzustellen. 
	Hierfür gibt es ein Cmd-Tool welches über die 
	

	
%\subsection{BOM - Bill of Matirial}
	
\subsection{Kostenplan}
 Durch die geringen Kosten pro Entwicklungsstufe hatten wir genügend Geld um mehrere Iterationsstufen  zu durchlaufen.
 Deswegen benötigten wir für die gesamte Entwicklung etwas über hundert Euro.
 Ein einzelnes Modul kann für etwa 45\,\euro\ nachgebaut werden. 
 Die Kosten setzen sich nach Tabelle\,\ref{Kosten für eine Giessanlage} zusammen.
 
 Nicht berücksichtigt sind das Wassergefäß, die zwei \begin{math}4mm\end{math}\,-Schlauch"-stücke und Befestigung in der Pflanze.
 Bei diesen handelt es sich um Reste oder Lagerfunde die von Wert von unter 1\,\euro\ sind. 
 
\begin{table}[h]
	\centering
	\onehalfspacing
	\footnotesize
	\caption{Kosten für eine Gießanlage}
	\label{Kosten für eine Giessanlage}
		\begin{tabular}{|l|ll|}
			\hline
\textit{Bauteil} & \textit{Kosten} & \textit{Bezugsquelle} \\
\hline
Arduino Nachbau & ca. 3 \euro & ebay \\
LCD-Display & ca. 5 \euro & ebay\\
Bodensensoren & ca. 2 \euro & ebay \\
Zahnradpumpe & 2,95 \euro & Pollin \\
XBee &  23,55 \euro & Reichelt \\
Hauptplatine & ca. 6 \euro & FabLAB \\
Gehäuse	& ca. 4 \euro & FabLAB \\

\hline
Gesamt: & ca 45 \euro & \\
\hline
\end{tabular}
\end{table}


%section

\section{Sparversion}
	Um Strom zu sparen, die Größe zu minimieren und Kosten zu reduzieren, haben wir die \glqq Sparversion\grqq entwickelt.
	In dieser Variante der Gießanlage wurde die Anlage auf die wesentlichen Komponenten beschränkt.
	Sie wurde so konzipiert, dass sie genau auf das Gießen einer Pflanze zugeschnitten ist.
	Sie kann nur durch erneutes Programmieren auf andere Pflanzen und Böden angelernt werden. 	
	\subsection{Aufbau}
	In dieser Version wird auf das Display und die Kommunikation verzichtet.
	Dadurch wird ein Großteil der Verkabelung eingespart, außerdem lässt sich die Hauptplatine deutlich kleiner gestalten.
	Dies führt zu geringerem Platz"-bedarf des Systems und folglich zu geringeren Gehäuseabmessungen.
	\subsection{Elektronik}
	Durch das Wegfallen der XBee Platine und deren Beschaltung wird das 3,3\,V Netz nicht mehr benötigt.
	Die Größe der Hauptplatine reduziert sich auf \begin{math} 55 mm \times 32 mm \end{math}.
	Dies entspricht weniger als der Hälfte der Fläche der Version 1.1 mit \begin{math} 56 mm \times 65 mm \end{math}.
	\subsection{Programmierung}
	Um dem Stromverbrauch weiter zu senken wurde diese Version nicht mit Arduino, sondern mit C geschriebenen Programm programmiert.
	Dies ermöglicht die Nutzung der Sleep"~Modi und der Interrupts des ATMega328.
	Dadurch befindet sich der \glqq Arduino Nano\grqq \ hauptsächlich im Schlafmodus und verbraucht deutlich weniger Energie.
	Die Gießeinstellungen müssen auf Grund der fehlenden Kommunikation und Eingabemöglichkeiten bei der Programmierung festgelegt werden.
	\subsubsection{Programmierwerkzeuge}
	Durch das Wegfallen der Arduino"~Bootloaders, kann die Hardware nicht mehr per USB programmiert werden. 
	Mit Hilfe des AVRISP mkII \footnote{\href{http://www.atmel.com/tools/avrispmkii.aspx}{www.atmel.com/tools/avrispmkii.aspx}} kann der Mikrocontroller über die ISP-Schnittstelle mit den Binärcode beschrieben werden. 

	\subsubsection{Logik}
		\begin{figure}[!h]
	\centering
	\includegraphics[width=0.8\linewidth]{Diagramme/SV_Ablaufdiagramm.png}
	\caption{Ablaufplan des Programms der Sparversion}
	\label{fig-SV_Ablaufplan}
\end{figure}

	Der Ablaufplan (Abbildung \ref{fig-SV_Ablaufplan}) der Sparversion wurde vereinfacht. Das Programm läuft linear immer wieder durch.
	Die meiste Zeit verbringt das System in der \glqq Sleep"~Schleife\grqq, in dem nach jedem Timer"~Überlauf der Zähler um eins dekrementiert wird.
	Nach ungefähr drei Stunden ist der Zähler klein genug und das Programm fährt die Sensoren hoch, indem er sie über den Transistor mit Strom versorgt.
	Dann erfolgt die Messung der Sensorwerte.
	Anhand der Sensorwerte wird nun entschieden, ob der Bedarf besteht zu Gießen.
	Nachdem Gießen wird der Zähler wieder hochgestellt und das Programm beginnt von vorne.
	\subsubsection{Kalbrierung der Sensoren}
	Es wurde auch berücksichtigt, dass der Bodensensor für jede Pflanze bzw. jeden Boden und teilweise jede Einstich"-stelle im Boden neu kalibriert werden muss.
	Es wurde entschieden dies manuell durchzuführen. 
	Dazu wird der Sensor in den trockenen Boden gesteckt und über die Vorschaltung mit Strom versorgt.
	Die Spannungsquelle an der Vorschaltung wird auf 5\,V eingestellt und die Spannung zwischen analogen Ausgang und Grund mit einem Multimeter gemessen.
	Nun wird solange gegossen bis die Erde feucht genug ist. Die gemessene Spannung wird notiert.
	Mit Hilfe der Formel \begin{math} { \frac{Messwert}{5,0\,V} } \times 1023 \end{math} wird der Wert errechnet, der in das Programm in die Definiton \emph{\#define FEUCHTE} eingetragen wird.
	
	
	
	\subsection{Kostenplan}

	\begin{table}[!h]
		\centering
		\onehalfspacing
		\footnotesize
		\caption{Kosten für eine  Sparversion Gießanlage}
		\label{Kosten für eine Sparversion Giesanlage}
	\begin{tabular}{|l|ll|}
			\hline
		\textit{Bauteil} & \textit{Kosten} & \textit{Bezugsquelle} \\
		\hline
		Arduino Nachbau & ca. 3 \euro & ebay \\
		Bodensensoren & ca. 2 \euro & ebay \\
		Zahnradpumpe & 2,95 \euro & Pollin \\
		Gehäuse	& 1,50 \euro & Pollin \\
		Hauptplatine & ca. 5,5 \euro & FabLAB \\
		\hline
		Gesamt: & ca 15 \euro & \\
		\hline
	\end{tabular}
	\end{table}
	
	  	
	Allein auf Grund des fehlenden XBee-Moduls halbiert sich der Preis der Anlage.
	Dazu kommt das nicht vorhandene Display, das kleinere Gehäuse und die günstigere Hauptplatine.
	So betragen die Kosten der Sparversion ca. 15 \euro\. 
	Einen Überblick über die respektiven Kosten gibt Tabelle \ref{Kosten für eine Sparversion Giesanlage}.
	 

	


\section{Resüme Do's and Dont's}

\end{document}


