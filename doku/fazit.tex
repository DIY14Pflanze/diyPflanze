
\section{Weiterentwicklung}
\subsection{Konfigurationstool}
Um die Daten der Gießanlage auslesen und die Konfiguration auf eine Pflanze vornehmen zu können, fehlt noch das PC"~Programm. 
Dieses Programm soll über eine Oberfläche die Einstellungen der Gießanlage(n) anzeigen und diese auf die jeweilige Pflanze, Boden- und Licht"-verhältnisse anpassen können.
Es gibt erste Ansätze, die jedoch noch am Anfang ihrer Entwicklung stehen.

\subsection{Kapazitive Bodenfeuchtigkeitsmessung}
Das größte Problem vor dem wir stehen ist der sich durch Elektrolyse zersetzende Bodenfeuchtigkeitssensor.
Durch das seltenere Messen und das Umpolen lässt sich die Lebensdauer des Sensor deutlich verlängern, aber nicht aufhalten.
So nimmt die Leitfähigkeit des Sensors mit der Zeit ab und die Messwerte veringern sich.
Außerdem besteht der Sensor aus einer verzinkten Kupferplatine und durch die Korrosion werden Kupferionen frei gesetzt. 
Diese werden von den Pflanzen aufgenommen, falls es sich bei den um Nutzpflanzen zum Verzehr handelt gelangen die giftigen Kupferionen in den menschlichen Körper.
 